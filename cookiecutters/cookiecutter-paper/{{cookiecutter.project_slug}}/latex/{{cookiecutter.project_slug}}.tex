\documentclass[final, authoryear, 5p, times]{elsarticle}
%\documentclass[final, 12pt, review, authoryear, times]{elsarticle}
\newcommand{\mytitle}{Paper title}
\newcommand{\myjournal}{Journal}
\newcommand{\myauthors}{X. Author, Y. Author and Z. Author}

\title{\mytitle}
\journal{\myjournal}

\usepackage[fleqn]{amsmath}
\setlength{\mathindent}{5pt}

\usepackage{amssymb}
\usepackage{amsthm}
\usepackage{bm}

\usepackage[breaklinks=true, colorlinks=true, linktocpage=true, pdftitle={\mytitle}, pdfauthor={\myauthors}, pdfsubject={Preprint submitted to \myjournal}, urlcolor=blue]{hyperref}

\usepackage[notref, notcite]{showkeys}
\renewcommand{\showkeyslabelformat}[1]{\color{red}\normalfont\scriptsize\ttfamily#1}

\usepackage{xcolor}

% 

\newcommand{\app}{\text{app}}
\DeclareMathOperator*{\argmin}{arg\,min}
\DeclareMathOperator{\conj}{conj}
\newcommand{\crit}{\text{crit}}
\newcommand{\D}{\mathrm d}
\newcommand{\dbldot}{\mathbin{\mathord{:}}}
\DeclareMathOperator{\dev}{dev}
\let\div\undefined
\DeclareMathOperator{\div}{div}
\newcommand{\E}{\mathrm e}
\newcommand{\eff}{\mathrm{eff}}
\DeclareMathOperator{\ensavg}{\mathbb E}
\DeclareMathOperator{\grad}{\mathbf{grad}}
\DeclareMathOperator{\HS}{HS}
\newcommand{\I}{\mathrm i}
\newcommand{\integers}{\mathbb Z}
\DeclareMathOperator{\iso}{\mathbf{iso}}
\newcommand{\jump}[1]{[\![#1]\!]}
\newcommand{\M}{\mathrm{m}}
\newcommand{\per}{\mathrm{per}}
\newcommand{\quaddot}{\mathbin{\mathord{::}}}
\newcommand{\reals}{\mathbb R}
\DeclareMathOperator{\sinc}{sinc}
\DeclareMathOperator{\sph}{sph}
\newcommand{\strains}{\mathcal E}
\newcommand{\stresses}{\mathcal S}
\DeclareMathOperator{\sym}{\mathbf{sym}}
\newcommand{\tens}[1]{\vec{#1}}
\newcommand{\tensors}{\mathcal T}
\newcommand{\todo}[1]{\color{red}TODO~---~#1\color{black}}
\newcommand{\tripledot}{\therefore}
\renewcommand{\vec}[1]{\bm{\mathrm{#1}}}
\DeclareMathOperator{\vol}{vol}

% 

\bibliographystyle{elsarticle-harv}

\begin{document}
\begin{frontmatter}
  \author[adr:1, adr:2]{X. Dupont}\ead{x.dupont@institution.fr}
  \author[adr:1]{S. Brisard\corref{cor:1}}\ead{sebastien.brisard@univ-eiffel.fr}
  \address[adr:1]{Navier, Ecole des Ponts, Univ Gustave Eiffel, IFSTTAR, CNRS,
    Marne-la-Vall\'ee, France}
  \address[adr:2]{Institution}

  \begin{abstract}
    This is truly an amazing paper
  \end{abstract}
  \begin{keyword}
    KW1\sep
    KW2\sep
    KW3\sep
    KW4
  \end{keyword}
\end{frontmatter}

\section{Introduction}

\citet{kani2003}.

\section{Preliminaries}

\begin{equation}
  \langle\tens\varepsilon\rangle=\frac1V\int_\Omega\tens\varepsilon\,\D V
\end{equation}

\begin{equation}
  \langle\tens\sigma\rangle=\tens C^\eff\dbldot\langle\tens\varepsilon\rangle
\end{equation}

\bibliography{{'{'}}{{cookiecutter.project_slug}}{{'}'}}

\end{document}

%%% Local Variables:
%%% coding: iso-latin-1
%%% fill-column: 80
%%% mode: latex
%%% TeX-master: t
%%% End:
